\documentclass[12pt,
bibtotoc,liststotoc,appendixprefix
twoside,paper=a4,headings=small]{scrbook}
%
% Packages
% -----------------------------------
\usepackage[
  paper=a4paper,
  left=12.5mm,
  right=25mm,
  top=25mm,
  bottom=50mm,
  bindingoffset=10mm]{geometry}		% Seitenränder und Bindungskorrektur einstellen
  
\usepackage{apacite} 				% Literatur-Referenzen: American Psycholog. Assoc.
\usepackage{natbib}					

\setcitestyle{round,aysep={}} 		% Indizierg. in runden Klammern, zw. Autor u. Jahr
\usepackage[utf8]{inputenc} 		% Umlaute im Text
\usepackage{ngerman}				% Rechtschreibg.
\usepackage[T1]{fontenc}
\usepackage{lmodern}				% Schriftfamilie
\usepackage{microtype}				% für die Mikrotypografie (besseres Schriftbild)

\usepackage{graphicx} 				% Grafiken einfügen (pdf,png - aber jpg vermeiden)
\graphicspath{{./Bilder/}}          % Pfad zu den Bildern

\usepackage{url}					% URL's formatieren (z.B. in Literatur) 
\usepackage[colorlinks,linkcolor=black,citecolor=black,urlcolor=black]{hyperref} 				% für Hyperlinks in PDF-Dokumenten   
  
\usepackage{tabularx} 				% bessere Gestaltung von Tabellen
\usepackage{longtable} 		
\usepackage{multicol}				
\usepackage{multirow}
\usepackage{booktabs}
\usepackage{tabularx}
		
\usepackage[active]{srcltx}

\usepackage{listings}				% Algorithmen

\usepackage{mdwlist}				% Listen

\usepackage{setspace} 				% Zeileneinstellung
\newtheorem{mydef}{Merksatz}  		% Falls Beispiele, Merksätze m. fortl. Nr. gebr. werden
\newtheorem{bsp}{Beispiel}

\usepackage{todonotes}				% zum Erstellen von ToDos im Editor

\usepackage{lscape}					% zum Rotieren von Seiten

\usepackage{amsmath}				% zum Schreiben von mathematischen Formeln

\usepackage{calc}

\usepackage{footnote}				% Fußnoten
\usepackage{tablefootnote}			% Fußnoten in Tabellen

%\clubpenalty = 10000
%\widowpenalty = 10000 \displaywidowpenalty = 10000

\hyphenation{voll-st\"andigen}		% Worttrennungen global definieren

\setcounter{tocdepth}{1}			% Ebenen, die im Inhaltsverzeichnis angezeigt werden

% Document
% -----------------------------------
\begin{document}

\frontmatter 
    % Titelseite soll keine Kopf oder Fußzeile haben
\thispagestyle{empty}

% Alle Elemente sollen zentriert sein
\begin{center}

\vspace*{-10mm}

\begin{figure}[h]
	\centering
		\includegraphics[width=0.20\textwidth]{./Bilder/fu_logo.pdf}
\end{figure}

\vspace*{-10mm}

{\LARGE PROFESSUR FÜR \\ANGEWANDTE STATISTIK\\[1mm]}
DER FREIEN UNIVERSITÄT BERLIN\\

\vspace*{3cm}


% Art der Arbeit => (Bachelorarbeit ,Diplomarbeit, Masterarbeit, Seminararbeit)
{\Large \textbf{Seminararbeit}}\\ 

\vspace{1cm}

% Titel der Arbeit 
{\Large \textbf{Topic models}}\\ 
\vspace*{1mm}
{\Large \textbf{applied to song texts}}\\ 
\vspace*{1mm}
% {\Large \textbf{Zeilen verteilt werden wenn nötig}}\\

\vspace{3.5cm}

% Name des/der Autors/Autoren
{\LARGE Silvia Ventoruzzo}\\[15mm]

% Gutachter, Kontaktdaten und Abgabetermin
\parbox{120mm}{
\begin{large}
\begin{tabbing}
Gutachter(in): \hspace{.7cm} \=Prof. Dr. Timo Schmid\\[4mm]
Kurs:\>Seminar zu angewandten Statistik\\
Semester:\> Sommersemester 2019\\
%Verfasser:\> Vorname Nachname\\ % alphabetische Reihenfolge (Nachname)
Matrikel-Nr.:\> 1234567\\
%Adresse:\> Straße Nr, PLZ Ort\\
%Email:\> mail@mail.de\\
%Telefon:\>030-123 456 78\\
[8mm]
\textbf{Abgabetermin:} \> \textbf{01. Januar 2099}\\
\end{tabbing}
\end{large}
}

\end{center}
\clearpage{\pagestyle{empty}\cleardoublepage}
 			% Titelblatt
    \newpage
    \clearpage{\pagestyle{empty}\cleardoublepage}
    \onehalfspacing                  	% Zeilenabstand ab hier 1.5 zeilig
    \tableofcontents 					% Inhaltsverzeichnis
    \clearpage{\pagestyle{empty}\cleardoublepage} 
    
    \listoffigures 					 	% Abbildungsverzeichnis
    \clearpage{\pagestyle{empty}\cleardoublepage}
    
    \listoftables						% Tabellenverzeichnis rein
    \clearpage{\pagestyle{empty}\cleardoublepage}
% -----------------------------------
\mainmatter 							% die einzelnen Kapitel
    \chapter{Einleitung}

Vestibulum fringilla pede sit amet augue. In turpis. Pellentesque posuere. Praesent turpis. Aenean posuere, tortor sed cursus feugiat, nunc augue blandit nunc, eu sollicitudin urna dolor sagittis lacus. Donec elit libero, sodales nec, volutpat a, suscipit non, turpis. Nullam sagittis. Suspendisse pulvinar, augue ac venenatis condimentum, sem libero volutpat nibh, nec pellentesque velit pede quis nunc. Vestibulum ante ipsum primis in faucibus orci luctus et ultrices posuere cubilia Curae; Fusce id purus. Ut varius tincidunt libero. Phasellus dolor. Maecenas vestibulum mollis diam. Pellentesque ut neque. Pellentesque habitant morbi tristique senectus et netus et malesuada fames ac turpis egestas. In dui magna, posuere eget, vestibulum et, tempor auctor, justo. In ac felis quis tortor malesuada pretium. Pellentesque auctor neque nec urna. Proin sapien ipsum, porta a, auctor quis, euismod ut, mi. Aenean viverra rhoncus pede. Pellentesque habitant morbi tristique senectus et netus et malesuada fames ac turpis egestas. Ut non enim eleifend felis pretium feugiat. Vivamus quis mi. Phasellus a est. Phasellus magna. In hac habitasse platea dictumst. Curabitur at lacus ac velit ornare lobortis. Curabitur a felis in nunc fringilla tristique.




\section{Problemumfeld}

Vestibulum fringilla pede sit amet augue. In turpis. Pellentesque posuere. Praesent turpis. Aenean posuere, tortor sed cursus feugiat, nunc augue blandit nunc, eu sollicitudin urna dolor sagittis lacus. Donec elit libero, sodales nec, volutpat a, suscipit non, turpis. Nullam sagittis. Suspendisse pulvinar, augue ac venenatis condimentum, sem libero volutpat nibh, nec pellentesque velit pede quis nunc. Vestibulum ante ipsum primis in faucibus orci luctus et ultrices posuere cubilia Curae; Fusce id purus. Ut varius tincidunt libero. Phasellus dolor. Maecenas vestibulum mollis diam. Pellentesque ut neque. Pellentesque habitant morbi tristique senectus et netus et malesuada fames ac turpis egestas. In dui magna, posuere eget, vestibulum et, tempor auctor, justo. 
\begin{figure}
	\centering
		\includegraphics[width=0.7\textwidth]{./Bilder/bsp1.png}
	\caption{Beispiel 1 zum Einfügen einer Grafik}
	\label{fig:bsp1}
\end{figure}

In ac felis quis tortor malesuada pretium. Pellentesque auctor neque nec urna. Proin sapien ipsum, porta a, auctor quis, euismod ut, mi. Aenean viverra rhoncus pede. Pellentesque habitant morbi tristique senectus et netus et malesuada fames ac turpis egestas. Ut non enim eleifend felis pretium feugiat. Vivamus quis mi. Phasellus a est. Phasellus magna. In hac habitasse platea dictumst. Curabitur at lacus ac velit ornare lobortis. Curabitur a felis in nunc fringilla tristique. 

\section{Zielsetzung}

Vestibulum fringilla pede sit amet augue. In turpis. Pellentesque posuere. Praesent turpis. Aenean posuere, tortor sed cursus feugiat, nunc augue blandit nunc, eu sollicitudin urna dolor sagittis lacus. Donec elit libero, sodales nec, volutpat a, suscipit non, turpis. Nullam sagittis. Suspendisse pulvinar, augue ac venenatis condimentum, sem libero volutpat nibh, nec pellentesque velit pede quis nunc. Vestibulum ante ipsum primis in faucibus orci luctus et ultrices posuere cubilia Curae; Fusce id purus. Ut varius tincidunt libero. Phasellus dolor. Maecenas vestibulum mollis diam. Pellentesque ut neque. Pellentesque habitant morbi tristique senectus et netus et malesuada fames ac turpis egestas. In dui magna, posuere eget, vestibulum et, tempor auctor, justo. In ac felis quis tortor malesuada pretium. Pellentesque auctor neque nec urna. Proin sapien ipsum, porta a, auctor quis, euismod ut, mi. Aenean viverra rhoncus pede. Pellentesque habitant morbi tristique senectus et netus et malesuada fames ac turpis egestas. Ut non enim eleifend felis pretium feugiat. Vivamus quis mi. Phasellus a est. Phasellus magna. In hac habitasse platea dictumst. Curabitur at lacus ac velit ornare lobortis. Curabitur a felis in nunc fringilla tristique.

\section{Aufbau der Arbeit}

Vestibulum fringilla pede sit amet augue. In turpis. Pellentesque posuere. Praesent turpis. Aenean posuere, tortor sed cursus feugiat, nunc augue blandit nunc, eu sollicitudin urna dolor sagittis lacus. Donec elit libero, sodales nec, volutpat a, suscipit non, turpis. Nullam sagittis. Suspendisse pulvinar, augue ac venenatis condimentum, sem libero volutpat nibh, nec pellentesque velit pede quis nunc. Vestibulum ante ipsum primis in faucibus orci luctus et ultrices posuere cubilia Curae; Fusce id purus. Ut varius tincidunt libero. Phasellus dolor. Maecenas vestibulum mollis diam. Pellentesque ut neque. Pellentesque habitant morbi tristique senectus et netus et malesuada fames ac turpis egestas. In dui magna, posuere eget, vestibulum et, tempor auctor, justo. In ac felis quis tortor malesuada pretium. Pellentesque auctor neque nec urna. Proin sapien ipsum, porta a, auctor quis, euismod ut, mi. Aenean viverra rhoncus pede. Pellentesque habitant morbi tristique senectus et netus et malesuada fames ac turpis egestas. Ut non enim eleifend felis pretium feugiat. Vivamus quis mi. Phasellus a est. Phasellus magna. In hac habitasse platea dictumst. Curabitur at lacus ac velit ornare lobortis. Curabitur a felis in nunc fringilla tristique. 


    \clearpage{\pagestyle{empty}\cleardoublepage}		% löscht Kopfzeilen und Seitennummerierung von der letzten Seite eines Kapitels, sofern dort kein Text mehr steht
    \chapter{Theory}

Topic models are generative probabilistic models used to ..... [FONTE]. Latent Dirichlet Allocation (LDA) belongs to this category and is specifically a "Bayesian mixture model for discrete data where topics are assumed to be uncorrelated" \citep{hornik2011topicmodels}.

\section{Terminology}
LDA is mainly used in the field of text mining, even though it can be applied to other fields \citep{blei2003latent}. Therefore it is important to firstly explain the RELATIVE terms, as defined by \cite{blei2003latent}:
\begin{itemize}
\item Corpus: a collection of M documents - $D = \{w_1, ..., w_M\}$
\item Document: a sequence of N words - $w = (w_1, ..., w_N)$
\item Word: item from a vocabulary - $w \{1, ..., V\}$
\end{itemize}

It follows that, in looking for the topic distribution for a document in a corpus, one .... following coefficients:
\begin{itemize}
\item $\beta$: term distribution of a topic
\item $\theta$: the proportion of the topic distribution for a document
\end{itemize}


\section{Model definition}
When generating a document $w$ of N words from a $V$-long vocablary from a corpus $D$, LDA follows three steps [FONTE: topicmodels JSS]:
\begin{enumerate}
\item 
\item Document: a sequence of N words - $w = (w_1, ..., w_N)$
\item Word: item from a vocabulary - $w \{1, ..., V\}$
\end{enumerate}

    \chapter{Anwendung}



\section{Daten}



\section{Analyse}


    \clearpage{\pagestyle{empty}\cleardoublepage}
    \input{./Kapitel/ergebnisse}
    \clearpage{\pagestyle{empty}\cleardoublepage}
    \input{./Kapitel/fazit}
    \clearpage{\pagestyle{empty}\cleardoublepage}

% -----------------------------------
\backmatter 
\bibliographystyle{apacite}				% bei natbib in deutsch
\bibliography{./Literatur/quellen}		% Literaturquellen einbinden 
\newpage
\thispagestyle{empty}

\begin{large}

\vspace*{2cm}

\noindent
Hiermit erkläre ich an Eides statt, dass ich die vorliegende Arbeit selbständig
und ohne unerlaubte fremde Hilfe angefertigt, andere als die
angegebenen Quellen und Hilfsmittel nicht benutzt und die den benutzten
Quellen und Hilfsmitteln wörtlich oder inhaltlich entnommenen Stellen als
solche kenntlich gemacht habe.

\vspace{2cm}

\noindent
Berlin, den 16. Februar 2011

\vspace{3cm}

\hspace*{7cm}%
\dotfill\\
\hspace*{8.5cm}%
\textit{(Unterschrift des Verfassers)}

\end{large}
 			% Eidesstattliche Erklärung (nicht bei Seminararb.)

\end{document}